% Dans l'introduction, on présente le problème étudié et les buts
% poursuivis. L'introduction permet de faire connaître le cadre de la
% recherche et d'en préciser le domaine d'application. Elle fournit
% les précisions nécessaires en ce qui concerne le contexte de
% réalisation de la recherche, l'approche envisagée, l'évolution de
% la réalisation. En fait, l'introduction présente au lecteur ce
% qu'il doit savoir pour comprendre la recherche et en connaître la
% portée.
% !TEX root = Document.tex.

\Chapter{INTRODUCTION}\label{sec:Introduction}  % 10-12 lignes pour introduire le sujet.

Les requis d'entreposage sont une partie intégrante des coûts associés à différents modes de transport. La deuxième moitié du 20\textsuperscript{ième} siècle a vu une augmentation marquée de l'utilisation de l'automobile, dont l'entreposage est très gourmand en espace. Pour la plupart des juridictions, la solution fut le développement des requis minimums de stationnement. Ces requis avaient pour but de faire internaliser aux développeurs, habitants, commerçants et employeurs les coûts associés à la mise à disposition d'entreposage pour le moyen de transport de leurs visiteurs, plutôt que de demander au contribuable de directement payer pour un stationnement public \parencite{shoup_high_2005}.\par

Cette rationnelle bien intentionnée a eu de nombreuses conséquences malencontreuses. En essayant de rendre le stationnement gratuit et disponible en tout point, la plupart des juridictions nord-américaines ont créé entre 2 et 14 places de stationnement pour chaque automobile \parencite{scharnhorst_quantified_2018}. Cette offre de stationnement abondante et gratuite à l'utilisation constitue une subvention chiffrée aux alentours de 5000\$ par année-véhicule \parencite{litman_comprehensive_2023} payée sous forme de loyers,  de biens et services et de dommages environnementaux.\par

Malgré ces coûts élevés, il n'existe pas d'inventaire complet des places de stationnements pour les municipalités québécoises, constituant un frein autant à la compréhension des enjeux qu'à l'établissement de politiques chiffrées permettant de mieux arbitrer les besoins d'entreposage de véhicules privés avec d'autres objectifs sociétaux comme la réduction des coûts totaux de transport, l'équité et la réduction des émissions de gaz à effet de serre. \par



%%
%%  CONCEPTS DE BASE / BASIC CONCEPTS
%%
%\section{Définitions et concepts de base}\label{sec:Definitions} % environ 2-3 pages

%Plusieurs types de stationnement sont mis à disposition des automobilistes dépendamment des circonstances et de l'aménagement du territoire. Il est nécessaire d'expliciter ces types, car chacun requerra potentiellement une méthodologie différente pour en faire l'inventaire. Cette section sera sous-divisée en deux parties : une première expliquant les morphologies physiques des différents types de stationnement et une deuxième expliquant les modalités de mise à disposition (prix, règlementation, permis, horaire).

%\subsection{Typologies de stationnements}

%\paragraph{Stationnement sur rue} Ce type de stationnement est mis à disposition des automobilistes en bordure de la voirie publique. Son accès peut être régi par plage horaire (pour maximiser la capacité routière, opérer au nettoyage ou au déneigement de la voirie), par permis (pour résidents par exemple), par tarif (comme c'est le car pour les artères commerçantes pour maximiser le roulement) ou peut être gratuit et disponible en tout temps.

%\paragraph{Stationnement hors rue en terrain vague} Ce type de stationnement existe principalement de façon informelle ou un propriétaire foncier met à disposition un terrain avec très peu d'œuvres pour accommoder les automobilistes. Il est souvent gratuit, mais peut être tarifé et utilisé comme capacité de débordement là où la demande est suffisante. 

%\paragraph{Stationnement hors rue en surface} Ce type de stationnement est le plus commun en périphérie des grands centres urbains. Il s'agit d'un terrain asphalté et ligné, typiquement attenant un bâtiment, mais peut être un espace dédié.

%\paragraph{Stationnement hors rue à étages} Ce type de stationnement est une structure multiétage, lignée permettant d'accommoder plus d'automobilistes à une surface donnée. Ce sont souvent des structures dont l'usage est dédié à l'entreposage de véhicules.

%\paragraph{Stationnement hors rue souterrain} Ce type de stationnement est construit en sous-sol souvent en combinaison avec un bâtiment résidentiel ou commercial pour permettre d'entreposer les véhicules avec une demande en terrain minimale.

%\subsection{Modalités d'accès au stationnement}

%L'accès au stationnement peut être régi de différentes manières. Cette section présentera les modalités présentement mises en place dans la région de Montréal pour différents types de stationnement.

%\paragraph{Stationnement gratuit - interdiction horaire périodique} La plupart du stationnement sur rue à Montréal est disponible gratuitement pour l'usager avec une période horaire par semaine où l'usager doit déplacer son véhicule. Les modalités sont le plus souvent indiquées sur un panneau explicatif en bordure de rue.

%\subsection{Stationnement tarifé par horodateur} Ce type de stationnement utilise une borne à proximité ou un paiement en ligne associé à un numéro de place ou une plaque d'immatriculation pour arbitrer l'accès aux places de stationnement. Ce type de stationnement est populaire dans les endroits où il y a une forte demande pour du stationnement sur rue, comme dans les artères commerçantes, pour des stationnements en surface proches du centre-ville. Ce type de paiement est typiquement contrôlé par une agence de sécurité. Dans le contexte montréalais, l'agence de mobilité durable est responsable de la gestion du stationnement.

%\subsection{Stationnement tarifé à l'accès} Ce type de stationnement limite l'accès au stationnement en demandant le paiement en entrée ou sortie du stationnement. Il est typiquement présent pour des structures de stationnement dédiées comme les stationnements à étages ou souterrains. Le taux peut être horaire

%\subsection{Stationnement à accès limité par vignette ou permis} Ce type de stationnement utilise un permis attribué par une autorité gouvernementale ou un propriétaire foncier pour limiter l'accès au stationnement. La plupart du temps administrée au moyen d'une vignette ou d'un dispositif affiché sur le véhicule en question, l'accès par permis peut aussi être administré en limitant l'accès au moyen d'une barrière. La ville de Montréal dispose d'un système de vignettes qui sont attribuées aux résidents d'un secteur moyennant un paiement, mais il peut être attribué comme droit acquis en fonction d'un statut(résidence, emploi, handicap, etc).


%\clearpage

%%
%% ELEMENTS DE LA PROBLEMATIQUE
%%
\section{Éléments de la problématique}\label{sec:Problematique}  % environ 3 pages
La ville de Québec est en voie de se doter d'une politique de stationnement, une étape franchie pour la ville de Montréal en 2016 \parencite{ville_de_montreal_politique_2016}. Cela étant dit, il n'existe à l'heure actuelle pas d'inventaire de la quantité de stationnement disponible, de sa tarification et de la variation temporelle de cette offre. Cette section présentera certains des enjeux relatifs à la mise en place d'un tel inventaire et reprend des éléments de \parencite{bourdeau_methodologie_2014} qui a tenté un inventaire similaire pour la région de Montréal.\par

\subsection{Variabilité de l'objet}
Malgré les définitions posées à la section \ref{sec:typologie_stationnement}, l'objet stationnement peut varier énormément dans le temps et l'espace. D'une part, l'augmentation de la taille des voitures \parencite{pineau_tendances_2023} va réduire le nombre de véhicules qu'il est possible d'entreposer à superficie constante. De l'autre, les variations de capacité dans le temps, du fait de l'interdiction réglementaire, de la création de stationnements impromptus sur des terrains privés pour gérer des pointes ou du fait de l'accumulation de débris peuvent aussi faire varier les quantités de stationnement. 


\subsection{Données}

L'un des principaux problèmes à la mise en place d'un inventaire relate aux données. Des données sur l'offre de stationnement existent sous forme directe (plans de constructions, comptages) ou indirectes (panneaux de stationnements, données sur les bornes fontaines et les entrées charretières, codes d'urbanisme). Des données sur l'utilisation de stationnement existent aussi sous forme directe(relevés de transactions de parcomètres) ou indirectes(résultats d'enquête OD).\par
La disponibilité, la complétude, la précision et l'hétérogénéité des formats forment des barrières à l'exploitation des données. D'autre part, des besoins variables d'agrégation pour différentes utilisations de la donnée rendent une procédure universelle de traitement et de stockage difficile à élaborer. Un portrait juste et adapté aux besoins d'une intervention, soit elle politique, de terrain ou académique, est donc difficile à rendre.

\subsection{Enjeux}

\textcite{bourdeau_methodologie_2014} mentionne les enjeux reliés à la provision de  stationnement et ses effets sur la mobilité. À cela s'ajoutent trois formes d'arbitrage; une dimension spatiale entre différentes utilisation du sol qui sont en compétition pour un espace limité, particulièrement dans les milieux urbains denses avec un cadre bâti patrimonial et des besoins croissants sur ce même espace ; la deuxième est politique où l'offre de stationnement est perçue comme un enjeu de taille dans l'espace politique \parencite{mattioli_political_2020,button_political_2006}, sans réellement que le débat ne soit cadré par des données ; le troisième est économique où la provision de stationnement par mandat gouvernemental réduit la viabilité financière de projets de construction du fait de la quantité de terre et de construction supplémentaire imposée par ces mandats \parencite{cutter_parking_2012,jung_who_2011}. De plus, la littérature indique que la densité tend à réduire les coûts de services pour les municipalité \parencite{garrido-jimenez_municipal_2018,metro_vancouver_regional_planning_cost_2023,blais_perverse_2024}, même si certains sont en désaccord \parencite{ladd_population_1992,windsor_critique_1979}.
%%% 
% ETAT FINAL DESIRE
%%%%%


%\section{L'outil d'analyse du stationnement idéal}
%Cette section va décrire une méthodologie d'analyse qui pourrait être utilisé par un praticien pour évaluer le stationnement, ces effets sur la mobilité et les externalités qui y sont reliés. Deux grands types d'analyses sont possible: un diagnostic de la situation actuelle et l'évaluation de changements de politiques sur la mobilité des personnes et des marchandises ainsi que les externalités qui y sont reliées. Ce mémoire n'a pas pour but de faire l'ensemble de l'analyse mais cette section veut établir un cadre d'analyse global et illustrer les chaînons manquants actuels et où il contribuera à cet ensemble global.\par

%\hl{!!!!! À COMPLÉTER !!!!!}
    %\subsection{Indicateurs pertinents}

    %\subsection{Diagnostique de l'état actuel}
    %    \subsubsection{Étapes du diagnostic}
    %    \subsubsection{Évaluation de l'offre}
    %\paragraph{Estimation de l'offre sur rue}
    %\paragraph{Analyse de la règlementation passée}
    %\paragraph{Estimation de la capacité par méthode règlementaire}
    %\paragraph{Estimation de la capacité par inventaire direct}
    %\paragraph{Estimation de la capacité par imagerie satellite}
    %\paragraph{Consolidation d'une estimation de capacité à partir des différentes %méthodes d'estimation}
    %    \subsubsection{Évaluation de la demande}
    %\paragraph{Estimation de la demande à partir des enquêtes OD}
    %\paragraph{Comptage manuel de véhicule sur le terrain}
    %\paragraph{Comptage automatisé des véhicules}
    %  \subparagraph{Comptage par caméra sur véhicule gouvernemental}
    %  \subparagraph{Comptage analyse d'image aéroportée}
    %  \subparagraph{Comptage à partir de caméra de sécurité}
    %  \subparagraph{Ajout de capteurs aux places de stationnement}
    %    \subsubsection{Évaluation de l'utilisation de la capacité}
    %    \subsubsection{Influence de l'utilisation de la capacité sur la mobilité des %personnes}
    %    \subsubsection{Évaluation des coûts de remplacement et d'opportunité de l'état actuel}
    %    \subsubsection{Diagramme de la procédure de diagnostique}
    %\subsection{Évaluation de politiques futures}
    %    \subsubsection{Élaboration de scénarios}
    %    \subsubsection{Évaluation de l'effet des scénarios sur l'offre de %stationnement}
    %    \subsubsection{Évaluation de la variation de l'offre sur la mobilité des personnes}
    %    \subsubsection{Évaluation des indicateurs de la mobilité sur l'effet}
    %    \subsubsection{Diagramme de la méthodologie d'analyse}


%%
%% OBJECTIFS DE RECHERCHE / RESEARCH OBJECTIVES
%%
\section{Objectifs de recherche}  \label{sec:obj_recherche}% 0.5 page
Cette thèse aura pour but de développer un système d'information capable de générer, stocker, valider et disséminer aux parties prenantes des estimés sur la capacité, la localisation et la tarification de places de stationnement hors-rue d'un territoire. Les sous-objectifs suivants sont poursuivis :

\begin{enumerate}
\item Développer une structure de données qui permet de représenter les règlements de zonage portant sur la quantité de stationnement à fournir
\item Développer une méthode d'inventaire du stationnement hors-rue basée sur la réglementation du point 1 et les données de rôle foncier.
\item Définir une structure de base de données qui permet d'inventorier le stationnement et de représenter visuellement la capacité de stationnement sur le territoire
\item Valider la d'estimations contre une réalité terrain pour assurer la significativité statistique des estimés sur une variété de contextes géographiques et d'utilisations du sol
\item Développer un interface facile d'utilisation qui permet à un praticien d'évaluer la capacité et de corriger les estimés au besoin
\item Développer la structure d'un interface de programmation d'application pour que d'autres logiciels puissent valoriser la donnée dans les études de la mobilité
\end{enumerate}

%%
%% PLAN DU MEMOIRE / THESIS OUTLINE
%%
\section{Plan du mémoire}  % 0.5 page

Le mémoire se divisera en 5 chapitres. Le chapitre \ref{sec:Introduction} est une introduction qui présente le sujet, ses problématiques et la pertinence de l'étude. Le chapitre \ref{sec:RevLitt} est une revue de littérature sur les méthodes d'inventaire de places de stationnement, des coûts associés à la provision de différents types de stationnements hors-rue et aux effets de la variation de l'offre sur la mobilité des personnes. Le chapitre \ref{sec:Methodologie} portera sur les données disponibles et les méthodologies utilisées pour faire l'inventaire de l'ensemble des places de stationnement disponibles. Le chapitre \ref{sec:Resultats} présentera l'outil développé et l'inventaire complété pour le stationnement dans la capitale nationale. Le chapitre \ref{sec:Conclusion} présentera une synthèse des travaux, les limitations de la méthode utilisée et de potentielles améliorations futures.

\clearpage
