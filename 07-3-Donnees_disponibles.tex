\section{Données de départ}\label{sec:meth_donnees_dispo}
  \subsection{\ac{SIG}}
    \subsubsection{Synthèse des données disponibles}
      Le tableau \ref{tab:donnees_disponibles_Québec} résume les données disponibles pour la ville de Québec et leur source.
      \begin{landscape}
        \LTcapwidth=\textwidth
      \begin{longtable}[h!]{p{.2 \linewidth} p{.1 \linewidth} p{.3 \linewidth} p{.15\linewidth} p{.125\linewidth} }
        
        
        \hline
        Géobase & Type & Description  & Source & Date téléchargement.\\ 
        \hline
        \hline
        \endhead
        \hline
        \endfoot
        \hline
        \caption{Géobases pour le territoire de la ville de Québec}
        \label{tab:donnees_disponibles_Québec}
        \endlastfoot
        vdq-panneaux stationnement    & Points        & Panneaux de stationnement sur rue          & Ville de Québec / Données Québec  & 5 mai 2024 \\
        \hline
        vq\_quartiers & Polygones & SIG des quartiers de la ville de Québec & Ville de Québec / Données Québec & 5 mai 2024 \\
        \hline
        vdq-bornesfontaines          & Points        & Borne-fontaine sur rue                   & Ville de Québec / Données Québec & 12 mai 2024 \\
        \hline
        vq\_reseau\_routier\_2023 & Polylignes    & Bords de voiries pour circulation automobile  & Ville de Québec / Geo-index  & 12 juin 2023\\ 
        \hline
        vq\_stationnement\_2021  & Polylignes    & Bords des aires de stationnement hors rue & Ville de Québec / Geo-index & 8 mai 2021\\
        \hline
        vdq\_voie\_publique            & Lignes        & Centres de chaussées, trottoirs séparés et pistes cyclables & Ville de Québec / Données Québec & 16 avril 2024 \\
        \hline
        vdq-zonage-grille.xlsx          & Tableau        & Usages autorisés et classification (urbain/structurant/général) & Ville de Québec / Données Québec & 8 juin 2024 \\
        \hline
        vdq-zonagemunicipalzones          & Polygones        & Unités de voisinages selon le zonage municipal & Ville de Québec / Données Québec & 6 mai 2024 \\
        \hline
        vdq\_intersection\_voie \_publique & Points & Intersections avec les dispositifs de contrôle & Ville de Québec / Données Québec & 8 mai 2024 \\
        \hline
        vdq\_quartiers & Polygones &Séparation de la ville en quartiers & Ville de Québec / Données Québec & 8 mai 2024\\
        \hline
        Usages prédominants 2023  & Polygones & Usages prédominants du sol &   Ministère des Affaires municipales et de l'Habitation & 21 mai 2024 \\
        \hline
        Arrêts bus et à vélo 2024 & Points et polylignes & Arrêts de bus, parcours des lignes et bornes vélopartage & Réseau de transport de la capitale & 31 mai 2023 \\
        \hline
        Année de construction des chaussées & Polylignes & Année de construction des rues & Ville de Québec / Géoindex & 28 mai 2024 \\
        \hline
        Rôle foncier & Entrées de tableaux & Données en format xml du rôle foncier & Ministère des Affaires municipales et de l'Habitation & 28 mai 2024 \\
        \hline
        Rôle foncier géobase & FGDB (Points + tables) & Données SIG du rôle foncier & Ministère des Affaires municipales et de l'Habitation & 28 mai 2024 \\
        \hline
        highway & Polylignes & Centres de chaussées & \ac{OSM} & 13 mai 2024\\
        \hline
        parking & points et polygones & Stationnement recensés dans \ac{OSM} & \ac{OSM} & 2 mai 2024 \\
        \hline
          parking entrance & Points & Entrées de stationnement sous-terrains  & \ac{OSM} & 2 mai 2024 \\
          \hline
        
      \end{longtable}

      \begin{table}[h!]
        \centering
        \begin{tabular}{p{0.18\linewidth} | p{0.1\linewidth} | p{0.3\linewidth} | p{0.3\linewidth}} 
        \hline
        Nom du champs & Type Contenu & Description  & Exemple\\ 
        \hline
        ID             & Entier    & Entier identifiant unique pour chaque panneau  & 370758 \\ 
        & & & \\
        TYPE\_CODE      & Texte     & Code d'identification de chaque type de panneau de stationnement & PP1016\\
        & & & \\
        DESCRIPTION     & Texte     & Texte imprimé sur le panneau & Stat. int. 16h-18h LUN À VEN (fl. dou.)\\
        \hline
        \end{tabular}
        \caption{Champs de la géobase de panneaux de stationnement de la ville de Québec \parencite{ville_de_quebec_panneaux_2024}}
        \label{tab:champs_geobase_stationnement_quebec}
      \end{table}
      \end{landscape}
    \subsubsection{Illustration - Cas 1: Coin Gomin / Marguerite-Bourgeoys / Laurier}
    La section suivante va donner un aperçu de quelques intersections typiques pour illustrer les données disponibles. Le but principal est d'illustrer les enjeux.
      Les figures \ref{fig:donnes_panneaux_Laurier} et \ref{fig:donnes_polygone_panneaux_Laurier} donnent un aperçu des données disponibles pour l'intersection nommée ci-dessus. On constate que plusieurs panneaux sur un même poteau ne sont pas représentés géographiquement au même endroit. D'autre part, la présence de panonceaux donne des exceptions aux limitations de temps de stationnement. De plus, la ville a à sa disposition une géobase de bords de rue, mais aucune information n'est disponible pour associer le bord de rue à un tronçon donné. Le même constat est possible pour les panneaux de stationnement dont les seules informations sont un identifiant, la description et un identifiant de panneau. Ils ne sont pas associés à un tronçon ou un côté de rue. L'une des conséquences de la dispersion des panneaux est aussi la difficulté d'assigner les panneaux à un tronçon aux intersections puisque le panneau peut être plus proche d'un bord de rue autre du fait du décalage des panneaux dans l'espace. Dans ce cas-ci, le panneau Stat. int. (fl. ga.) est à 6m de la rue Marguerite Bourgeoys et 9m de la rue Gomin.
      \begin{figure}[ht]
        \centering
        \begin{subfigure}{\linewidth}
          \includegraphics[width=1.0\textwidth]{images/donnees_disponible_Laurier_legende_v2.png}
        \caption{Données linéaires disponibles}
        \label{fig:donnes_panneaux_Laurier}
        \end{subfigure} \\
        \begin{subfigure}{\linewidth}
          \includegraphics[width=1.0\textwidth]{images/utilisation_sols_Laurier_v2.png}
        \caption{Données polygonales disponibles}
        \label{fig:donnes_polygone_panneaux_Laurier}
        \end{subfigure}
        \caption{Données disponibles intersection Laurier}
      \end{figure}

      \FloatBarrier
    \subsubsection{Illustration - Cas 2: Coin de la Canardières - Desroches}
    Les figures \ref{fig:donnes_panneaux_Desroches} et \ref{fig:donnes_polygone_panneaux_Desroches} illustrent les mêmes données à un autre coin de rue. Ici, la localisation des panneaux porte encore plus à confusion puisque les panneaux de 2 tronçons de rue sont quasiment superposés. Il sera donc difficile d'assigner les panneaux aux bords de rues automatiquement lorsque ces derniers sont aux abords des intersections.
      \begin{figure}[ht]
        \centering
        \begin{subfigure}{\linewidth}
          \centering
          \includegraphics[width=0.8\textwidth]{images/donnees_disponible_Desroches_legende_v2.png}
        \caption{Données linéaires disponibles}
        \label{fig:donnes_panneaux_Desroches}
        \end{subfigure} \\
        \begin{subfigure}{\linewidth}
          \centering
          \includegraphics[width=0.8\textwidth]{images/utilisation_sols_Desroches_v2.png}
        \caption{Données polygonales disponibles}
        \label{fig:donnes_polygone_panneaux_Desroches}
        \end{subfigure}
        \caption{Données disponibles intersection Desroches / De la Canardière}
      \end{figure}
    
  \subsection{Entrainement d'apprentissage Machine}
    Plusieurs ensembles de données ont été créés pour la détection de places ouvertes dans un stationnement. Plus récemment, un ensemble de données annotées a été créé pour 
  \subsection{Enquête OD}
  \subsection{Imagerie Aérienne}
  \subsection{Rôle Foncier}
  \subsection{Recensement}