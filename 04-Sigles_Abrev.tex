% Liste des sigles et abbréviations / List of symbols and acronyms
\ifthenelse{\equal{\Langue}{english}}{
	\newcommand\abbrevname{LIST OF SYMBOLS AND ACRONYMS}
}{
	\newcommand\abbrevname{LISTE DES SIGLES ET ABRÉVIATIONS}
}
\chapter*{\abbrevname}
\addcontentsline{toc}{compteur}{\abbrevname}
\pagestyle{pagenumber}
%
\begin{acronym}
  \acro{SIG}{Système d'Information Géographique}
  \acro{GPS}{Global Positionning System}
  \acro{CNN}{Convolutionnal Neural Networks}
  \acro{OSM}{OpenStreetMap}
  \acro{TC}{Transport en Commun}
  \acro{PLUTO}{Primary Land Use Tax Lot Output}
  \acro{IoU}{Intersection over Union}
  \acro{mIoU}{mean Intersection over Union}
  \acro{ReLu}{Unité de rectification linéaire ou \og{Rectified Linear Unit} \fg{}}
  \acro{OQLF}{Office Québecois de la Langue Française}
  \acro{CDS}{Coefficient de Dice Sorensen}
  \acro{CUBF}{Code d'Utilisation du Bien Fond}
  \acro{RST}{RegulationSetTerritory ou ensemble de règlements de territoire}
  \acro{PR}{ParkingRegulation ou règlement de stationnement}
  \acro{PRS}{ParkingRegulationSet ou ensemble de règlement}
  \acro{TD}{TaxDataset ou ensemble de données foncières}
  \acro{PI}{ParkingInventory ou inventaire de stationnement}
  \acro{PAV}{Profils d'Accumulation de Véhicules}
  \acro{API}{Application Programming Interface ou interface de Programmation applicative}
  \acro{HTTP}{HyperText Transfer Protocol}
  \acro{REST}{Representational State Transfer}
  \acro{URL}{Uniform Resource Locator}
  \acro{URI}{Uniform Resource Identifier}
  \acro{TC}{Transport Collectif}
  \acro{ESRI}{Environmental Systems Research Institute, Inc.}
\end{acronym}
%
\begin{longtable}{lp{5in}}
SIG     & Système d'Information Géographique                                            \\
GPS     & Global Positionning System                                                    \\
CNN		& Convolutionnal Neural Networks                                                \\
OSM     & OpenStreetMap                                                                 \\
TC      & Transport en Commun                                                           \\
PLUTO   & Primary Land Use Tax Lot Output                                               \\
IoU     & Intersection over Union ou indice de Jaccard                                  \\
mIoU    & mean Intersection over Union ou indice de Jaccard moyen                       \\
ReLu    & Unité de rectification linéaire ou \og{Rectified Linear Unit} \fg{}           \\
OQLF    & Office Québécois de la Langue Française                                       \\
CDS     & Coefficient de Dice Sorensen                                                  \\
CUBF    & Code d'Utilisation des Biens-Fonds                                            \\
RST     & RegulationSetTerritory ou Ensembles de Règlements de Territoire               \\
PAV     & Profils d'Accumulation de Véhicules                                           \\
PR      & ParkingRegulation ou Règlement de Stationnement                                \\
PRS     & ParkingRegulationSet ou Ensemble de Règlements de Stationnement               \\
TD      & TaxDataset ou Ensemble de données foncières                                   \\
API     & Application Programming Interface ou interface de Programmation applicative   \\
HTTP    & HyperText Transfer Protocol                                                   \\
REST    & REpresentational State Transfer                                               \\
URL     & Uniform Resource Locator                                                      \\
URI     & Uniform Resource Identifier                                                   \\
ESRI    & Environmental Systems Research Institute, Inc. \\
$F_{men}$ & Facteur de pondération ménage dans l'enquête OD                             \\
$v_{men}$ & Nombre de véhicules d'un ménage dans l'enquête OD                           \\
$T_{men}$ & Variable binaire qui indique la première entrée d'un ménage dans la table de déplacements de l'enquête OD \\
$F_{per}$ & Facteur de pondération d'une personne dans l'enquête OD                     \\
$T_{per}$ & Variable binaire qui indique la première entrée d'une personne dans la table de déplacements de l'enquête OD\\
$PP_{per}$ & Variable catégorielle dénotant la possession du permis de conduire: 1 = Oui, 2 = Non, 3= Refus/NSP, 4=Non applicable\\
$M_{dep}$ & Variable catégorielle pour le mode du déplacement 1 = Auto-Conducteur, 2 = Auto-Passager, 5 = Vélo, 6 = TC, 7=Bus Scolaire, 13= À pied, 14= Traversier avec véhicule, 15 = Traversier sans véhicule\\
$H$ & Heure de départ d'un déplacement de l'enquête OD \\ 
$X_{ori}$ & Longitude de l'origine d'un déplacement dans l'enquête OD \\
$Y_{ori}$ & Latitude de l'origine d'un déplacement dans l'enquête OD\\
$P_{ori}$ & Position de l'origine d'un déplacement dans l'enquête OD. Utilisé pour dénoter les intersections spatiales \\ 
$X_{log}$ & Longitude du lieu de résidence dans l'enquête OD\\
$Y_{log}$ & Latitude du lieu de résidence dans l'enquête OD\\
$P_{log}$ & Position du lieu de résidence dans l'enquête OD. Utilisé pour dénoter les intersections spatiales \\
$X_{des}$ & Longitude de la destination d'un déplacement dans l'enquête OD\\
$Y_{des}$ & Latitude de la destination d'un déplacement dans l'enquête OD\\
$P_{des}$ & Position de la destination d'un déplacement dans l'enquête OD. Utilisé pour dénoter les intersections spatiales \\
$P_{role}$ & Position (Point) d'une entrée du rôle foncier \\
$p_{AD}$ & Population d'une aire de diffusion\\
$P_q$ & Population du quartier q\\
$PT_{q}$ & Nombre de détenteurs de permis de conduire pour le quartier q\\
$V_q$ & Nombre de véhicule pour le quartier q\\
$S_q$ & Nombre de places de stationnement du quartier q\\
$s_l$ & Nombre de places de stationnement d'un lot\\
$G_l$ & Géométrie d'un lot cadastral \\
$G_q$ & Géométrie d'un quartier\\
$G_{AD}$ & Géométrie d'une aire de diffusion\\
$A(G)$ & Aire de la géométrie G\\
$D_s$ & Densité de stationnement pour un quartier \\
$d_s$ & Densite de stationnement pour un lot\\
$D_{f,T,q}$ & Densité du foncier total pour un quartier q\\
$D_{f,R,q}$ & Densité du foncier résidentiel pour un quartier q\\
$d_f$ & Densité du foncier pour un lot\\
$D_p$ & Densité de population pour un quartier\\
$O_{res}$ & Taux d'occupation de stationnement des résidents \\
$O_{max}$ & Taux d'occupation des stationnement maximal \\
$VS_{res}$ & Nombre de voiture des résidents par place de stationnement dans un secteur. Inverse du taux d'occupation\\ 
$VS_{max}$ & Nombre de voiture maximales par place de stationnement dans un secteur. Inverse du taux d'occupation\\
$a_l$ & Superficie d'un lot cadastral\\
$a_{etage}$ & Aire d'étage d'une entrée du rôle foncier \\
$n_{logements}$ & Nombre de logements dans une entrée du rôle foncier\\
$c_{}$
$A_q$ & Superficie d'un quartier\\
$f_l$ & Valeur foncière d'une entrée au rôle foncier\\
$F_{T,q}$ & Valeur foncière totale agrégée pour un quartier\\
$F_{R,q}$ & Valeur foncière résidentielle totale agrégée pour un quartier\\
$AS_q$ & Superficie des places de stationnement dans un quartier\\
$AS_{q,\%}$ & Pourcentage de la superficie du quartier dédiée au stationnement\\
$N_{auto,q,h}$ & Nombre d'automobiles à l'heure h dans le quartier q. Utilisé dans les profils d'accumulation véhicule\\
$N_{auto,ent,q,h}$ & Nombre d'automobiles entrant dans le quartier q à l'heure h\\
$N_{auto,sor,q,h}$ & Nombre d'automobiles sortant du quartier q à l'heure h \\
$PM_{mode}$ & Part modale du mode en indice. L'indice peut être AC (auto-conducteur), AP (auto-passager), TC (Transport en commun), MV (Marche Vélo) et BS (Bus scolaire) au fins de cette étude 
\end{longtable}

