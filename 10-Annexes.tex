%%
%%  Annexes
%%
%%  Note: Ne pas modifier la ligne ci-dessous. / Do not modify the following line.
\ifthenelse{\equal{\Langue}{english}}{
	\addcontentsline{toc}{compteur}{APPENDICES}
}{
	\addcontentsline{toc}{compteur}{ANNEXES}
}
%%
%%
%%  Toutes les annexes doivent être inclues dans ce document
%%  les unes à la suite des autres.
%%  All annexes must be included in this document one after the other.
\Annexe{Requêtes SQL pertinentes}
Cette section détaillera certaines requêtes qui n'ont pas été listées dans le texte principal du mémoire, mais qui sont pertinentes.
\section{Identification lots problématiques}\label{sec:annexe_identification_lots_problématiques}
Du fait de problèmes de complétude des données du rôle foncier, il est potentiellement nécessaire d'aider l'analyste à identifier des lots qui doivent potentiellement être revisités.
\begin{lstlisting}
    SELECT
    	cad.g_no_lot,
    	COUNT(rf.id_provinc) AS compte_role,
    	COUNT(
    		CASE 
    			WHEN (rf.rl0308a IS NULL AND rf.rl0105a::int <= 1999)
    				 OR (rf.rl0311a IS NULL AND rf.rl0105a::int >= 2000)
    			THEN 1 
    		END
    	) AS compte_donnee_manquant,
    	COUNT(CASE WHEN rf.rl0307a IS NULL THEN 1 END) AS compte_date_manquant,
    	(CASE WHEN inv.n_places_min = 0 THEN TRUE ELSE FALSE END) AS n_places_min_nul
    FROM role_foncier rf
    JOIN association_cadastre_role acr ON acr.id_provinc = rf.id_provinc
    JOIN cadastre cad ON acr.g_no_lot = cad.g_no_lot
    JOIN inventaire_stationnement inv ON cad.g_no_lot = inv.g_no_lot
    WHERE inv.methode_estime = 2
    GROUP BY cad.g_no_lot, inv.n_places_min;
\end{lstlisting}
\section{Inventaire compilé par quartier}\label{sec:annexe_inventaire_agrege_par_quartier}
L'agrégation du stationnement est faite avec une requête spatiale assez onéreuse. D'autre part, un lot aurait potentiellement plusieurs inventaires avec différentes méthodes. L'application web permet de prioriser différents types d'inventaires en fonction de la préférence de l'analyste. Les colonnes inv\_XYZ représentent chacune une priorité entre les différentes méthodologies d'inventaire.
\begin{lstlisting}
    DELETE FROM stat_agrege;
        WITH LotNeighborhood AS (
          SELECT 
            c.g_no_lot, 
            s.id_quartier,
            s.superf_quartier,
            s.geometry,
            s.nom_quartier
          FROM public.cadastre c
          JOIN public.sec_analyse s 
            ON ST_Area(ST_Intersection(s.geometry, c.geometry)) / ST_Area(c.geometry) > 0.9
        ),

        inv_123 AS (
          SELECT ln.id_quartier, ln.geometry, ln.superf_quartier, ln.nom_quartier,
                CEIL(SUM(COALESCE(sub.parking_estimate, 0))) AS inv_123
          FROM LotNeighborhood ln
          LEFT JOIN LATERAL (
            SELECT 
              COALESCE(
                CASE WHEN i.methode_estime = 1 THEN i.n_places_mesure
                    ELSE i.n_places_min
                END, 0
              ) AS parking_estimate
            FROM inventaire_stationnement i
            WHERE i.g_no_lot = ln.g_no_lot
            ORDER BY CASE i.methode_estime
                      WHEN 1 THEN 1
                      WHEN 2 THEN 2
                      WHEN 3 THEN 3
                      ELSE 4
                    END
            LIMIT 1
          ) sub ON true
          GROUP BY ln.id_quartier, ln.geometry, ln.superf_quartier,ln.nom_quartier
        ),

        inv_132 AS (
          SELECT ln.id_quartier, ln.geometry, ln.superf_quartier, 
                CEIL(SUM(COALESCE(sub.parking_estimate, 0))) AS inv_132
          FROM LotNeighborhood ln
          LEFT JOIN LATERAL (
            SELECT 
              COALESCE(
                CASE WHEN i.methode_estime = 1 THEN i.n_places_mesure
                    ELSE i.n_places_min
                END, 0
              ) AS parking_estimate
            FROM inventaire_stationnement i
            WHERE i.g_no_lot = ln.g_no_lot
            ORDER BY CASE i.methode_estime
                      WHEN 1 THEN 1
                      WHEN 3 THEN 2
                      WHEN 2 THEN 3
                      ELSE 4
                    END
            LIMIT 1
          ) sub ON true
          GROUP BY ln.id_quartier, ln.geometry, ln.superf_quartier
        ),

        inv_213 AS (
          SELECT ln.id_quartier, ln.geometry, ln.superf_quartier, 
                CEIL(SUM(COALESCE(sub.parking_estimate, 0))) AS inv_213
          FROM LotNeighborhood ln
          LEFT JOIN LATERAL (
            SELECT 
              COALESCE(
                CASE WHEN i.methode_estime = 2 THEN i.n_places_min
                    WHEN i.methode_estime = 1 THEN i.n_places_mesure
                    ELSE i.n_places_min
                END, 0
              ) AS parking_estimate
            FROM inventaire_stationnement i
            WHERE i.g_no_lot = ln.g_no_lot
            ORDER BY CASE i.methode_estime
                      WHEN 2 THEN 1
                      WHEN 1 THEN 2
                      WHEN 3 THEN 3
                      ELSE 4
                    END
            LIMIT 1
          ) sub ON true
          GROUP BY ln.id_quartier, ln.geometry, ln.superf_quartier
        ),

        inv_231 AS (
          SELECT ln.id_quartier, ln.geometry, ln.superf_quartier, 
                CEIL(SUM(COALESCE(sub.parking_estimate, 0))) AS inv_231
          FROM LotNeighborhood ln
          LEFT JOIN LATERAL (
            SELECT 
              COALESCE(
                CASE WHEN i.methode_estime = 2 THEN i.n_places_min
                    WHEN i.methode_estime = 1 THEN i.n_places_mesure
                    ELSE i.n_places_min
                END, 0
              ) AS parking_estimate
            FROM inventaire_stationnement i
            WHERE i.g_no_lot = ln.g_no_lot
            ORDER BY CASE i.methode_estime
                      WHEN 2 THEN 1
                      WHEN 3 THEN 2
                      WHEN 1 THEN 3
                      ELSE 4
                    END
            LIMIT 1
          ) sub ON true
          GROUP BY ln.id_quartier, ln.geometry, ln.superf_quartier
        ),

        inv_312 AS (
          SELECT ln.id_quartier, ln.geometry, ln.superf_quartier, 
                CEIL(SUM(COALESCE(sub.parking_estimate, 0))) AS inv_312
          FROM LotNeighborhood ln
          LEFT JOIN LATERAL (
            SELECT 
              COALESCE(
                CASE WHEN i.methode_estime = 3 THEN i.n_places_min
                    WHEN i.methode_estime = 1 THEN i.n_places_mesure
                    ELSE i.n_places_min
                END, 0
              ) AS parking_estimate
            FROM inventaire_stationnement i
            WHERE i.g_no_lot = ln.g_no_lot
            ORDER BY CASE i.methode_estime
                      WHEN 3 THEN 1
                      WHEN 1 THEN 2
                      WHEN 2 THEN 3
                      ELSE 4
                    END
            LIMIT 1
          ) sub ON true
          GROUP BY ln.id_quartier, ln.geometry, ln.superf_quartier
        ),

        inv_321 AS (
          SELECT ln.id_quartier, ln.geometry, ln.superf_quartier, 
                CEIL(SUM(COALESCE(sub.parking_estimate, 0))) AS inv_321
          FROM LotNeighborhood ln
          LEFT JOIN LATERAL (
            SELECT 
              COALESCE(
                CASE WHEN i.methode_estime = 3 THEN i.n_places_min
                    WHEN i.methode_estime = 2 THEN i.n_places_min
                    ELSE i.n_places_mesure
                END, 0
              ) AS parking_estimate
            FROM inventaire_stationnement i
            WHERE i.g_no_lot = ln.g_no_lot
            ORDER BY CASE i.methode_estime
                      WHEN 3 THEN 1
                      WHEN 2 THEN 2
                      WHEN 1 THEN 3
                      ELSE 4
                    END
            LIMIT 1
          ) sub ON true
          GROUP BY ln.id_quartier, ln.geometry, ln.superf_quartier
        )


        -- Final merge + insert
        INSERT INTO stat_agrege (
          id_quartier, geom, superf_quartier,nom_quartier,
          inv_123, inv_132, inv_213, inv_231, inv_312, inv_321
        )
        SELECT 
          i123.id_quartier,
          i123.geometry,
          i123.superf_quartier,
          i123.nom_quartier,
          i123.inv_123,
          i132.inv_132,
          i213.inv_213,
          i231.inv_231,
          i312.inv_312,
          i321.inv_321
        FROM inv_123 i123
        LEFT JOIN inv_132 i132 ON i123.id_quartier = i132.id_quartier
        LEFT JOIN inv_213 i213 ON i123.id_quartier = i213.id_quartier
        LEFT JOIN inv_231 i231 ON i123.id_quartier = i231.id_quartier
        LEFT JOIN inv_312 i312 ON i123.id_quartier = i312.id_quartier
        LEFT JOIN inv_321 i321 ON i123.id_quartier = i321.id_quartier;
\end{lstlisting}

\section{Nombre d'automobiles résidents et nombre maximal et minimal d'automobiles par quartier}\label{sec:annex_motorisation_par_quartier}
Le nombre de véhicules appartenant aux résidents d'un quartier peut être estimé à l'aide des données de l'enquête OD. Le nombre de véhicules des résidents est trouvé en multipliant le facteur ménage par le nombre de véhicules du ménages pour toutes les lignes qui sont en tête de ménage. Le nombre maximal de véhicules dans le quartier est issu des profiles d'accumulation véhicule \parencite{diallo_methodology_2015}.
\begin{lstlisting}
    DELETE FROM motorisation_par_quartier;
    INSERT INTO motorisation_par_quartier (id_quartier, nb_voitures,nb_voitures_max_pav,nb_voitures_min_pav,diff_max_signee)
    WITH aggregated_data AS (
        SELECT
            z.id_quartier::int,
            CEIL(SUM(c.nbveh * c.facmen)) AS nb_voitures
        FROM
            sec_analyse z
        JOIN
            od_data c
            ON ST_Within(c.geom_logis, z.geometry)
        WHERE
            c.tlog = 1
        GROUP BY
            z.id_quartier
    )
    SELECT
        ad.id_quartier,
        ad.nb_voitures,
        MAX(pav.voitures) AS nb_voitures_max_pav,
        MIN(pav.voitures) AS nb_voitures_min_pav,
        CASE
            WHEN ABS(ad.nb_voitures - MAX(pav.voitures)) > ABS(ad.nb_voitures - MIN(pav.voitures))
            THEN MAX(pav.voitures)-ad.nb_voitures
            ELSE MIN(pav.voitures) - ad.nb_voitures
        END AS diff_max_signee
    FROM
        aggregated_data ad
    LEFT JOIN
        profile_accumulation_vehicule pav
        ON pav.id_quartier::int = ad.id_quartier::int
    GROUP BY
        ad.id_quartier,
        ad.nb_voitures;
\end{lstlisting}
\section{Population issue du recensement}\label{sec:annexe_pop_quartiers}
Les données de recensement peuvent être agrégées par quartier en utilisant une approche similaire à au nombre d'automobile. Cependant, puisque les aires de diffusions sont des polygones et qu'il y a le potentiel d'avoir des aires de diffusions qui touchent à plusieurs quartiers, une condition a été ajouté voulant que 90\% de l'aire de diffusion doit être dans le quartier pour être sommée. 
\begin{lstlisting}
    DELETE FROM population_par_quartier;
    INSERT INTO population_par_quartier (id_quartier,pop_tot_2016,pop_tot_2021)
    WITH pop_2016_ag AS(
      SELECT 
        z.id_quartier,
        sum(c2016.pop_2016) as pop_tot_2016
      FROM
        sec_analyse z
      JOIN 
        census_population_2016 c2016
      ON 
        ST_Intersects(z.geometry,c2016.geometry)
      WHERE
        ST_Area(ST_Intersection(z.geometry,c2016.geometry))/ ST_Area(c2016.geometry) >= 0.9
      GROUP BY
      z.id_quartier
    ), pop_2021_ag as(
      SELECT
        z.id_quartier,
        SUM(c2021.pop_2021) AS pop_tot_2021
      FROM
        sec_analyse z
      JOIN
        census_population c2021
      ON
        ST_Intersects(z.geometry, c2021.geometry)
      WHERE
        ST_Area(ST_Intersection(z.geometry, c2021.geometry)) / ST_Area(c2021.geometry) >= 0.9
      GROUP BY
        z.id_quartier
    )
    SELECT
      c2016.id_quartier,
      c2016.pop_tot_2016,
      c2021.pop_tot_2021
    FROM
      pop_2016_ag c2016
    JOIN
      pop_2021_ag c2021
    ON
      c2016.id_quartier=c2021.id_quartier;
\end{lstlisting}
\section{Valeurs foncières agrégées}\label{sec:annexe_val_fonc_agreg}
Certaines valeurs foncières sont agrégées pour donner une indication du degré d'intensité d'utilisation du sol dans un quartier. Les valeurs sont agrégées pour le total mais aussi pour les logements seulement. La requête ci-dessous donne la méthode de calcul pour les donnnées foncières agrégées.
\begin{lstlisting}
    DELETE FROM donnees_foncieres_agregees;
    INSERT INTO donnees_foncieres_agregees (id_quartier,valeur_moyenne_logement,superf_moyenne_logement,valeur_fonciere_logement_totale,valeur_fonciere_totale)
    WITH role_quartier_log AS(
      SELECT 
        sa.id_quartier::int,
        ceil(SUM(rf.rl0404a)/sum(rf.rl0311a)) as val_moy_log,
        ceil(SUM(rf.rl0308a)/sum(rf.rl0311a)) as sup_moy_log,
        ceil(SUM(rf.rl0404a)) as val_tot_log
      FROM
        role_foncier rf
      LEFT JOIN
        sec_analyse sa on ST_Within(rf.geometry,sa.geometry)
      where
        rf.rl0105a::int = 1000 
        and rf.rl0404a IS not null 
        and rf.rl0308a IS not null 
        and rf.rl0311a IS not null
      group by sa.id_quartier
      
    ), role_quartier_tout AS(
      SELECT 
        sa.id_quartier::int,
        ceil(SUM(rf.rl0404a)) as val_tot_tout
      FROM
        role_foncier rf
      LEFT JOIN
        sec_analyse sa on ST_Within(rf.geometry,sa.geometry)
      where
        rf.rl0404a is not null 
      group by sa.id_quartier
    )
    SELECT 
      sa.id_quartier::int,
      rql.val_moy_log,
      rql.sup_moy_log,
      rql.val_tot_log,
      rqt.val_tot_tout
    from
      sec_analyse sa
    left join 
      role_quartier_log rql on rql.id_quartier = sa.id_quartier
    left join
      role_quartier_tout rqt on rqt.id_quartier = sa.id_quartier;
\end{lstlisting}

\section{Parts modales}\label{sec:annexe_parts_modales}
Cette section va donner la requête SQL qui compile les différentes parts modales définies dans la section \label{sec:meth_analyse}. 
\begin{lstlisting}
    DELETE FROM parts_modales;
    INSERT INTO parts_modales(id_quartier,ac_res,ap_res,tc_res,mv_res,bs_res,ac_ori,ap_ori,tc_ori,mv_ori,bs_ori,ac_des,ap_des,tc_des,mv_des,bs_des,ac_int,ap_int,tc_int,mv_int,bs_int)
    WITH trips AS (
        SELECT
            od.clepersonne,
            od.nolog,
            od.tlog,
            od.facmen,
            od.tper,
            od.mobil,
            od.facper,
            od.mode1,
            od.geom_logis,
            od.geom_ori,
            od.geom_des,
            sa_logis.id_quartier AS quartier_logis,
            sa_ori.id_quartier AS quartier_ori,
            sa_des.id_quartier AS quartier_des,
            CASE 
                WHEN sa_ori.id_quartier = sa_des.id_quartier AND sa_ori.id_quartier IS NOT NULL 
                THEN true ELSE false 
            END AS internal_trip
        FROM od_data AS od
        LEFT JOIN sec_analyse sa_logis ON ST_Within(od.geom_logis, sa_logis.geometry)
        LEFT JOIN sec_analyse sa_ori   ON ST_Within(od.geom_ori, sa_ori.geometry)
        LEFT JOIN sec_analyse sa_des   ON ST_Within(od.geom_des, sa_des.geometry)
        WHERE od.geom_ori IS NOT NULL AND sa_logis.id_quartier IS NOT NULL
    )
    SELECT
        sa.id_quartier,

        -- Resident trips
        SUM(CASE WHEN trips.quartier_logis = sa.id_quartier AND trips.mode1 = 1 THEN trips.facper END) / SUM(CASE WHEN trips.quartier_logis = sa.id_quartier THEN trips.facper END) * 100 AS AC_res,
        SUM(CASE WHEN trips.quartier_logis = sa.id_quartier AND trips.mode1 = 2 THEN trips.facper END) / SUM(CASE WHEN trips.quartier_logis = sa.id_quartier THEN trips.facper END) * 100 AS AP_res,
        SUM(CASE WHEN trips.quartier_logis = sa.id_quartier AND trips.mode1 = 6 THEN trips.facper END) / SUM(CASE WHEN trips.quartier_logis = sa.id_quartier THEN trips.facper END) * 100 AS TC_res,
        SUM(CASE WHEN trips.quartier_logis = sa.id_quartier AND trips.mode1 IN (5, 13) THEN trips.facper END) / SUM(CASE WHEN trips.quartier_logis = sa.id_quartier THEN trips.facper END) * 100 AS MV_res,
        SUM(CASE WHEN trips.quartier_logis = sa.id_quartier AND trips.mode1 = 7 THEN trips.facper END) / SUM(CASE WHEN trips.quartier_logis = sa.id_quartier THEN trips.facper END) * 100 AS BS_res,

        -- Origin trips (external)
        SUM(CASE WHEN trips.quartier_ori = sa.id_quartier AND trips.internal_trip = false AND trips.mode1 = 1 THEN trips.facper END) / SUM(CASE WHEN trips.quartier_ori = sa.id_quartier AND trips.internal_trip = false THEN trips.facper END) * 100 AS AC_ori,
        SUM(CASE WHEN trips.quartier_ori = sa.id_quartier AND trips.internal_trip = false AND trips.mode1 = 2 THEN trips.facper END) / SUM(CASE WHEN trips.quartier_ori = sa.id_quartier AND trips.internal_trip = false THEN trips.facper END) * 100 AS AP_ori,
        SUM(CASE WHEN trips.quartier_ori = sa.id_quartier AND trips.internal_trip = false AND trips.mode1 = 6 THEN trips.facper END) / SUM(CASE WHEN trips.quartier_ori = sa.id_quartier AND trips.internal_trip = false THEN trips.facper END) * 100 AS TC_ori,
        SUM(CASE WHEN trips.quartier_ori = sa.id_quartier AND trips.internal_trip = false AND trips.mode1 IN (5, 13) THEN trips.facper END) / SUM(CASE WHEN trips.quartier_ori = sa.id_quartier AND trips.internal_trip = false THEN trips.facper END) * 100 AS MV_ori,
        SUM(CASE WHEN trips.quartier_ori = sa.id_quartier AND trips.internal_trip = false AND trips.mode1 = 7 THEN trips.facper END) / SUM(CASE WHEN trips.quartier_ori = sa.id_quartier AND trips.internal_trip = false THEN trips.facper END) * 100 AS BS_ori,

        -- Destination trips (external)
        SUM(CASE WHEN trips.quartier_des = sa.id_quartier AND trips.internal_trip = false AND trips.mode1 = 1 THEN trips.facper END) / SUM(CASE WHEN trips.quartier_des = sa.id_quartier AND trips.internal_trip = false THEN trips.facper END) * 100 AS AC_des,
        SUM(CASE WHEN trips.quartier_des = sa.id_quartier AND trips.internal_trip = false AND trips.mode1 = 2 THEN trips.facper END) / SUM(CASE WHEN trips.quartier_des = sa.id_quartier AND trips.internal_trip = false THEN trips.facper END) * 100 AS AP_des,
        SUM(CASE WHEN trips.quartier_des = sa.id_quartier AND trips.internal_trip = false AND trips.mode1 = 6 THEN trips.facper END) / SUM(CASE WHEN trips.quartier_des = sa.id_quartier AND trips.internal_trip = false THEN trips.facper END) * 100 AS TC_des,
        SUM(CASE WHEN trips.quartier_des = sa.id_quartier AND trips.internal_trip = false AND trips.mode1 IN (5, 13) THEN trips.facper END) / SUM(CASE WHEN trips.quartier_des = sa.id_quartier AND trips.internal_trip = false THEN trips.facper END) * 100 AS MV_des,
        SUM(CASE WHEN trips.quartier_des = sa.id_quartier AND trips.internal_trip = false AND trips.mode1 = 7 THEN trips.facper END) / SUM(CASE WHEN trips.quartier_des = sa.id_quartier AND trips.internal_trip = false THEN trips.facper END) * 100 AS pBS_des,

        -- Internal trips
        SUM(CASE WHEN trips.quartier_ori = sa.id_quartier AND trips.internal_trip = true AND trips.mode1 = 1 THEN trips.facper END) / SUM(CASE WHEN trips.quartier_ori = sa.id_quartier AND trips.internal_trip = true THEN trips.facper END) * 100 AS AC_int,
        SUM(CASE WHEN trips.quartier_ori = sa.id_quartier AND trips.internal_trip = true AND trips.mode1 = 2 THEN trips.facper END) / SUM(CASE WHEN trips.quartier_ori = sa.id_quartier AND trips.internal_trip = true THEN trips.facper END) * 100 AS AP_int,
        SUM(CASE WHEN trips.quartier_ori = sa.id_quartier AND trips.internal_trip = true AND trips.mode1 = 6 THEN trips.facper END) / SUM(CASE WHEN trips.quartier_ori = sa.id_quartier AND trips.internal_trip = true THEN trips.facper END) * 100 AS TC_int,
        SUM(CASE WHEN trips.quartier_ori = sa.id_quartier AND trips.internal_trip = true AND trips.mode1 IN (5, 13) THEN trips.facper END) / SUM(CASE WHEN trips.quartier_ori = sa.id_quartier AND trips.internal_trip = true THEN trips.facper END) * 100 AS MV_int,
        SUM(CASE WHEN trips.quartier_ori = sa.id_quartier AND trips.internal_trip = true AND trips.mode1 = 7 THEN trips.facper END) / SUM(CASE WHEN trips.quartier_ori = sa.id_quartier AND trips.internal_trip = true THEN trips.facper END) * 100 AS BS_int
    FROM sec_analyse sa
    LEFT JOIN trips ON 
        trips.quartier_logis = sa.id_quartier OR 
        trips.quartier_ori = sa.id_quartier OR 
        trips.quartier_des = sa.id_quartier
    GROUP BY sa.id_quartier, sa.geometry;
\end{lstlisting}